%%
%% $Id$
%%
%% Authors:
%% 	Jeff Buchbinder <jeff@freemedsoftware.org>
%%
%% FreeMED Electronic Medical Record and Practice Management System
%% Copyright (C) 1999-2007 FreeMED Software Foundation
%%
%% This program is free software; you can redistribute it and/or modify
%% it under the terms of the GNU General Public License as published by
%% the Free Software Foundation; either version 2 of the License, or
%% (at your option) any later version.
%%
%% This program is distributed in the hope that it will be useful,
%% but WITHOUT ANY WARRANTY; without even the implied warranty of
%% MERCHANTABILITY or FITNESS FOR A PARTICULAR PURPOSE.  See the
%% GNU General Public License for more details.
%%
%% You should have received a copy of the GNU General Public License
%% along with this program; if not, write to the Free Software
%% Foundation, Inc., 675 Mass Ave, Cambridge, MA 02139, USA.
%%
%%	FreeMED TeX Template
%%	Progress Notes
%%

{[method namespace='org.freemedsoftware.api.PatientInterface.ToText' param0=$pnotespat param1=0 var='patientName']}
{[method namespace='org.freemedsoftware.module.ProviderModule.fullName' param0=$pnotesdoc param1=1 var='providerName']}
{[link link=$pnotesdoc table='physician' field='phypracname' var='providerPractice']}
{[link link=$pnotesdoc table='physician' field='phyaddr1a' var='providerAddress']}
{[link link=$pnotesdoc table='physician' field='phycitya' var='providerCity']}
{[link link=$pnotesdoc table='physician' field='phystatea' var='providerState']}
{[link link=$pnotesdoc table='physician' field='phyzipa' var='providerZip']}
{[link link=$pnotesdoc table='physician' field='phyphonea' var='providerPhone']}
{[link link=$pnotesdoc table='physician' field='phyfaxa' var='providerFax']}}

\documentclass{article}

% No normal header, footer, etc
% Extra room on bottom and left for bindings, etc
\usepackage[top=0.5in,left=1.2in,right=0.5in,bottom=1.2in,nohead,nofoot]{geometry}
% Relative size package
\usepackage{relsize}

% Macro for progress-note type headers
\newcommand{\sheading}[1]{\textbf{#1:}}

\begin{document}

% Deal with the document "header". For now, we're just looking at a one-time
% only display. FIXME: Migrate this to a header.
\begin{center}
        {\relsize{1}\textbf{{[$providerPractice|texize]}}} \\
        {\relsize{-1}\textsl{{[$providerName|texize]}}} \\
        \textsl{{[$providerAddress|texize]}, {[$providerCity|texize]}, {[$providerState|texize]} {[$providerZip|texize]} } \\
        \textsl{phone {[$providerPhone|texize]} fax {[$providerFax|texize]}}
\end{center}

\begin{tabular}{ll}
NAME:	&	{[$patientName|texize]}	\\
CHART:	&	{[link link=$pnotespat table='patient' field='ptid'|texize]} 	\\
DOB:	&	{[link link=$pnotespat table='patient' field='ptdob'|texize]}	\\
DATE:	&	{[$pnotesdt|texize]} \\
PRINTED: &	\today \\
\end{tabular}

\begin{center}
	{\relsize{1}\textbf{PATIENT PROGRESS NOTES}}
\end{center}

% Nice fixed block with \cr's in between them
\begin{tabular}{ll}
{[if $pnotesbmi]}
\sheading{BMI} & {[$pnotesbmi|texize]} \\
{[/if]}
{[if $pnotesheight]}
\sheading{Height} & {[$pnotesheight]} \\
{[/if]}
{[if $pnotesweight]}
\sheading{Weight} & {[$pnotesweight]} \\
{[/if]}
{[if $pnotessbp]}
\sheading{Blood Pressure} & {[$pnotessbp]} over {[$pnotesdbp]} \\
{[/if]}
{[if $pnotesheartrate]}
\sheading{Heart Rate} & {[$pnotesheartrate]} \\
{[/if]}
{[if $pnotesresprate]}
\sheading{Respiratory Rate} & {[$pnotesresprate]} \\
{[/if]}
{[if $pnotestemp]}
\sheading{Temperature} & {[$pnotestemp]} \\
{[/if]}
\  &  \  
\par
\end{tabular} \\

{[if strlen($pnotes_S) ge 10]}
\noindent\sheading{SUBJECTIVE} {[$pnotes_S|texize]}
\par
\medskip
{[/if]}

{[if strlen($pnotes_O) ge 10]}
\noindent\sheading{OBJECTIVE} {[$pnotes_O|texize]}
\par
\medskip
{[/if]}

{[if strlen($pnotes_A) ge 10]}
\noindent\sheading{ASSESSMENT} {[$pnotes_A|texize]}
\par
\medskip
{[/if]}

{[if strlen($pnotes_P) ge 10]}
\noindent\sheading{PLAN} {[$pnotes_P|texize]}
\par
\medskip
{[/if]}

{[if strlen($pnotes_I) ge 10]}
\noindent\sheading{INTERVAL} {[$pnotes_I|texize]}
\par
\medskip
{[/if]}

{[if strlen($pnotes_E) ge 10]}
\noindent\sheading{EDUCATION} {[$pnotes_E|texize]}
\par
\medskip
{[/if]}

{[if strlen($pnotes_R) ge 10]}
\noindent\sheading{RX} {[$pnotes_R|texize]}
\par
\medskip
{[/if]}

\end{document}

